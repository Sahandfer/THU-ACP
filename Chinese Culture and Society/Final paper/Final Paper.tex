\documentclass[12pt]{article}
\usepackage{xeCJK}
% \usepackage[utf8]{inputenc}
% \usepackage[english]{babel}
\usepackage[document]{ragged2e}
\usepackage{CJKutf8}
%
%Margin - 1 inch on all sides
%
\usepackage[letterpaper]{geometry}
\usepackage{times}
\geometry{top=1.0in, bottom=1.0in, left=1.0in, right=1.0in}

%
%Doublespacing
%
\usepackage{setspace}
\doublespacing

%
%Rotating tables (e.g. sideways when too long)
%
\usepackage{rotating}
%
%Fancy-header package to modify header/page numbering (insert last name)
%
\usepackage{fancyhdr}
\pagestyle{fancy}
\lhead{} 
\chead{} 
\rhead{Sabour \thepage} 
\lfoot{} 
\cfoot{} 
\rfoot{} 
\renewcommand{\headrulewidth}{0pt} 
\renewcommand{\footrulewidth}{0pt} 
%To make sure we actually have header 0.5in away from top edge
%12pt is one-sixth of an inch. Subtract this from 0.5in to get headsep value
\setlength\headsep{0.333in}


%
%Works cited environment
%(to start, use \begin{workscited...}, each entry preceded by \bibent)
% - from Ryan Alcock's MLA style file
%
\newcommand{\bibent}{\noindent \hangindent 40pt}
\newenvironment{workscited}{\newpage \begin{center} Works Cited \end{center}}{\newpage }


%
%Begin document
%
\begin{document}
	\begin{flushleft}
		
		%%%%First page name, class, etc
		Sahand Sabour (\begin{CJK*}{UTF8}{gbsn}
			山姆
		\end{CJK*} )\\
		2020280401\\
		Chinese Culture and Society (60610082-0)\\
		January 10 2020\\
		
		
		%%%%Title
		\begin{center}
			An Expository View of Chinese Feminism: Striving for Equal Rights
		\end{center}
		
		
		%%%%Changes paragraph indentation to 0.5in
		\setlength{\parindent}{0.5in}
		%%%%Begin body of paper here
		\justifying
		In many ways, the May Fourth Movement, which started in 1915, marked the beginning of a new era. This was an anti-imperialist, political, social and cultural movement that aimed to protest against the government’s decision on Japan’s territory over Shandong province. Numerous male intellectuals of the time broadened the movement’s aim step by step during the upcoming years to include new and different discussions. One of the many discussions was adopted from the western ideology about women’s rights and liberation, through which intellectuals strove to overcome the feudal barriers of inequality between men and women nested deep within Chinese culture and traditions. Hence, this period in Chinese history has been considered as a crucial point for the birth and the rise of Chinese feminism; one of the biggest events of the twentieth century that occurred on a global scale (Liu et al. 4). This paper briefly introduces the history of Chinese feminism, provides a rather thorough comparison between Western and Chinese ideologies towards feminism, and most notably, proposes an expository view on whether how feminism in China affected the lives of Chinese women and whether it was truly aimed at benefiting women’s lives and fighting for equal rights.
		
		Throughout the years, Chinese feminism has developed in various social and cultural aspects, and though there has yet to be a clear definition of it, it could be described by its different waves. In accordance to Yu’s terminology (308), as there are no clear transitions between the different periods and stages of Chinese feminism, similar to western feminism, they would be referred to as \textit{waves} (with no clear start or end) for the remainder of this paper. One of the initial contributions to the first wave of Chinese feminism, which was believed to have started in the May Fourth Movement, was the \textit{New Youth} published by Chen Duxiu in the January of 1916. In his work, Chen Duxiu formally criticized the three principles of Confucianism and proposed that women shall not be conquered, but rather become conquerors. This was not the first nor the last piece of criticism towards Confucianism; mainly due to the oppressive acts towards women, such as foot-binding (i.e. breaking and binding the feet of young girls as to change the shape and size of their feet) and concubinage, that were deeply embedded in Confucian principles. As differences of feminism between western and Chinese ideology are believed to stem from such differences in culture and beliefs, the root of this ideology must first be analyzed in order to provide a thorough comparison.
		
		\vspace{0.3cm}
		\noindent \textbf{Differences of Men and Women in Confucianism}
		
		\noindent From the early periods of Chinese history, the concept of yin and yang played an essential role in the ancient construct of Chinese philosophy. In short, these were concepts that were used to demonstrate how the universe functioned, representing the fact that each entity, given its independent state and relationship to others, fails to exist in isolation of other entities. The relation between night and day, light and dark, as well as heaven and earth are all believed to be instances of this phenomena in Chinese traditions. However, as mentioned by Yun (586-587), the applications of yin and yang were further extended in Confucian ideology to also be applied to male and female relations, in which men were associated with yang (positivity) and therefore, were granted with more power and social privilege over women and more importantly, the ability to dominate over all instances of yin (negativity), namely women. 
		
		The effect of such assumptions on society is evident even to this day as many women are judged by their bodily features and nature while being denied for positions of power, dominance and decision making. In Confucianism, women existed for men, to complement men’s external affairs by committing their own affairs within the family; women did not have the opportunity to truly reach their moral objective and be valued as individual entities outside the bounds of their families. In Chinese culture, men and women are distinguished by their societal roles rather as categories of nature (Brownell 25-26). Hence, Chinese feminism did not share roots with the Western topology of feminism in the sense that Western feminism was established upon scientific and biological constructs of men and women. This is an important factor in how Chinese feminism was defined throughout the twentieth century. 
		
		\vspace{0.3cm}
		\noindent \textbf{The First Wave of Chinese Feminism}
		
		\noindent 
		As results of the May Fourth Movement protests, inhumane and feudal customs such as foot-binding were denounced. Moreover, a new Marriage Law addressing traditional marriage arrangements and giving women the opportunity to choose their own marriage partners was established. In addition, women were provided the right to an education, similar to the men in their families. The denouncement of such oppressive acts as well as the provision of these rights were essential for and well deserved by Chinese women. However, despite the difference of western and Chinese ideologies in considering the meaning of men and women, which was previously mentioned, the feminism activists of the time adopted the western meaning of feminism during the first wave of the movement in China: the term was translated to nüquan zhuyi, meaning “women’s right-ism”, in order to reflect the political demands of the activists of the time (Xu 203). From a personal viewpoint, this would have been a reasonable and beneficial choice if given that China shared the same societal background and chose a similar approach for future development as western countries of the time. However, that was not the case and this might have been a significant reason as to why the feminist movement declined in the latter decades after its somewhat success during the first wave. In addition, as can be seen in the later decades of this movement, the main Chinese feminism activist, in opposed to the Western feminism movement, are men. Indeed, there were many female feminist who posed great ideologies and published influential pieces, some even long before the May Fourth Movement. However, the main changes and shifts in ideologies of this movement were part of the male activists’ contributions, which serves to show how the societal position of men was considerably higher than those of women. 
		
		As mentioned, women had gained more power over decisions regarding their marriage partners, thus leaving the parents with little to no choice in such decisions. However, as a recurring theme in the history of Chinese feminism, new problems tended to rise and replace the old dilemmas. The new marriage law was proven to be costly for rural families that had paid for an arranged marriage so that they could have have a bride as their helping hand; accordingly, some rural marriage traditions, mainly patrilocality (i.e. the process of moving brides from their original homes to the homes of their husband’s family), were still intact after the establishment of the Marriage Law (Hershatter 225). This is significant as it forced women to move out of their natal homes, where they were known and had obtained a social identity, to a place in which they were considered as strangers and had to once again, establish a social identity for themselves (Johnson 1983). Similarly, many of the women in such areas who practiced their right to a divorce were either denied a divorce, as the township believed the couple could revive the relationship, or more severely, committed suicide or were murdered by their family members (Huang 2005, 179). 
		
		\vspace{0.3cm}
		\noindent \textbf{The Second Wave of Chinese Feminism}
		
		\noindent 
		The second wave of Chinese feminism is believed to have occurred during the same period as the establishment of People’s Republic of China (PRC) (Yu 310), during which Chinese feminism transformed into a state policy as party-state authorities aimed to mobilize rural and urban women into the social neighborhoods and workplaces (Barlow 2001, 1288). In the early periods of this wave, public discussion focused on personal life and portrayals of feminine sexuality dominated magazines and advertisements (Hershatter 266). The ideal women were judged based on their attractiveness and were portrayed as obedient wives who had the right to engage in political affairs and practice their societal rights, equal to their husbands, but would rather sacrifice this right to contribute to the socialist agenda (Hershatter 231). 
		
		Feminism in this period, also referred to as the Mao era, had become a repulsive subject. The Maoist ideology considered the previously used term \textit{women’s right-ism} as suggestive towards a political agenda and therefore, a bourgeois term that should be reserved for western ideology (Wang 1997). Hence, upon adopting a Marxist-Leninist-Maoist ideology, the terms gender (xingbie or difference of sex) and feminism were redefined as \textit{social gender} (shehui xingbie) (Wang 11) and \textit{socialist feminism} (shehui zhuyi nüquan zhuyi) (Chen 278). The party believed that true liberation of women and their rights could only be made possible in a “socialist revolution with a top-down approach” (Yu 313). In an effort to provide equality, the sex difference of male and female did not carry the old meanings as women were inserted into roles and political positions that were previously occupied by men (Mann 49). Similarly, statements such as “Men and women are the same”, which were uttered by Mao himself, further highlighted the ideology that in order to for all individuals to have equal rights, they must also share similar societal positions and have equivalent contributions. This in turn gave rise to the term \textit{iron girls} (tie gu niang), which was used to describe the humble and hardworking middle to low class urban women that were considered as essential parts of the socialist movement (Yu 313). 
		
		At first glance, this perception may seem to suggest that socialism laid strong grounds for women’s liberation and equality between men and women. However, it could be argued that this was not the case. The social standards portrayed the traits of male masculinity and essentially, encouraged all individuals to be like men. Statements such as “Men and Women are the same” muttered by Mao himself were implemented in ways that androgenize women, disregard the natural differences of male and female sex as well as the actual meaning of equality since women’s interest were not considered to any extent. From a personal perspective, equality should provide all individuals with the same opportunities yet it should also enable them to choose how to tackle these opportunities based on their own interest. The latter part of this definition is believed to have been discarded from socialist ideology as women were mobilized to perform the same work that men were previously responsible for in addition to the tasks such as house keeping and caring for the internal family matters that were traditionally labelled as women’s responsibilities. Not only may these reforms not be considered as an effort for equality but they should be rather be regarded as oppressive (Wang 19). 
		The period of the Great Leap Forward further intensified this oppression, especially in rural areas, as women were expected to work long hours on the fields for production. It was said that many women suffered from exhaustion and in many severe cases, even experienced miscarriages and uterine prolapses (Gao 607) due to the high production demand. In addition, child care was not available for all women at the time. This was reported to have many fatal consequences as there were tales of children wandering off and drowning or getting attacked by animals when their mothers were away working on the fields to serve the socialist state (Hershatter 241). Moreover,  androgenizing women was not the only downside of Chinese feminism in Mao era. Domestic violence was not condemned in any way, shape or form. Rather it was normalized as a custom of “smacking the wife around” (Hershatter 232). Similarly, this inhumane act was not considered by authorities as a legitimate reason for divorce either. Having all these issues in mind, it becomes evident that the feminist movement in its second wave barely focused on women’s rights and liberation as they thought these issues are rather minor and would be resolved in time upon further economic and cultural development (Hershatter 244).
		
		\vspace{0.3cm}
		\noindent \textbf{The Third Wave of Chinese Feminism}
		
		\noindent 
		The third wave of Chinese feminism was made possible in the post-Mao era, in which topics such as domestic violence and sexual portrayals of women were the focus of the discussion (Yu 309). After adopting the post-Mao reform, the intellectual space for debating women’s rights and feminist activism, which was shut down in the Mao era (Ko and Wang 6), was reopened once again. Accordingly, more work on this topic was allowed to be published, which was also not granted access during Mao era. This in turn enabled Western theories to make a return into the ideologies surrounding Chinese feminism. In the post-Mao era, the term \textit{iron girls} had been converted into \textit{socialist housewives }(i.e. women who devoted their lives to the betterment of their families (Leung 365)), which was fairly similar to the Confucian ideology that was despised from the birth of the Chinese feminist movement. This further demonstrates the earlier notion and adds upon it: in order for Chinese feminism to serve its purpose and successfully attain its objectives, the meaning of it has to be established based on the difference of the two sexes while keeping in mind the existing social and cultural constructs in China. As previously described, each of the previous two waves of this movement focused on one of these fields while disregarding the other. Hence, based on the mentioned notion, it would have been reasonable for this movement not to achieve what it had planned or advertised for in its earlier stages. 
		
		However, during the third wave, female intellectuals and feminist activists began to highlight the sex differences between men and women by encouraging the use of the term \textit{second sex}. This was aimed to oppose the Maoist view that men and women are the same as that is not the case: consciousness of men and women are different from one another by nature (Yu 169), yet they deserve equal rights. In addition, works of Min was highly essential during this period, as they highlighted the significance of adopting Western ideologies in a way that is suitable for the Chinese context of Feminism, which was recognized as an important requirement for the success of Chinese feminism earlier in this paper. This led to the replacement of the initial term \textit{woman right-ism} to be replaced by \textit{woman-ism} to describe the current wave of Chinese feminism (Yu 314). Later on, the Marriage Law was also modified to include family violence as a legitimate reason for divorce, yet issues regarding domestic violence were not addressed nor denounced by the Courts and authorities. This is believed to be due to the fact the law, similar to the social state idea, valued the well-being and harmony of family higher then the well-being of individuals who experienced domestic abuse. 
		
		\vspace{0.3cm}
		\noindent \textbf{Chinese Feminism in the Current Era}
		
		\noindent 
		During the current era, which some consider as the fourth wave of Chinese feminism, the topic of this movement has moved beyond the difference of men and women to discuss topics of homosexuality, bisexuality, etc. (Yu 315). The aim of striving for equal rights has also been replaced by the call for equity (i.e. giving the oppressed population and minorities more opportunities in order for all share the same outcome and societal level upon seizing those opportunities) amongst different genders. This is believed to be adopted by the current definition of Western feminism, which is also geared towards similar topics. However, from a personal viewpoint, this context of feminism still does not fit the societal construct of modern China. In today’s society, many problems of the previous waves have yet to be addressed. For instance, the social position of women as expected by society, and sometimes even themselves, requires them to sacrifice their own desire and interest for the greater good of their families and societies. In addition, interactions and posts on huge social medias, such as Dou Yin, portray the notion that women should take care of the housework and internal affairs while their husbands work and earn money for them to spend on luxurious bags and clothes. This is believed to somewhat encourage the traditional Confucian belief that women exist as servants to men. It should mentioned that this argument does not aim to discourage women who enjoy this lifestyle but rather is a piece of criticism towards families and cultural standards that force women into this lifestyle. Hence, it is believed that these problems should be addressed prior to what is currently being discussed instead of directly adopting Western ideologies, as this approach had previously proven to be highly damaging to the main causes of this movement.
		
		\vspace{0.3cm}
		\noindent \textbf{Conclusion}
		
		\noindent Chinese feminism has undergone many adjustments from its supposed birth during the May Fourth Movement in the twentieth century to the current era. This paper briefly described the history and the evolution of Chinese feminism based on its four different waves. Accordingly, the societal background and ideology for each wave was provided and discussed. In addition, traditional beliefs of Chinese culture were briefly touched upon to provide a solid background for latter arguments. Comparisons of the Western and Chinese versions of feminism were also provided to delve deeper into the argument as to why adoption of Western ideas should be done with caution while having in mind the context of the newly adopted term in the cultural and societal environments of China. During each wave, the aims and accomplishments of the Chinese feminist movement at the time was demonstrated and its impact on the lives of Chinese women was thoroughly analyzed. It was argued that the cultural and societal differences of Western and Chinese societies still have a dominant effect on how the context of feminism should be defined in the modern Era, arguing the fact that feminism in China has occasionally been used to pose a political agenda and/or advertisement purposes rather than focusing on the well-being of Chinese women and striving for equal rights for all genders; however, it was not neglected that many of this movement’s contributions have actually changed Chinese women’s lives for the better. Feminism still remains an unclear yet essential topic in both China and Western countries and the fight for equality shall not be stopped. 
		
		
		
		
		%%%%Works cited
		\begin{workscited}
			\bibent Brownell, Susan, and Jeffrey N. Wasserstrom. 2002. \textit{Chinese Femininities / Chinese Masculinities.} Berkeley: University of California Press. ISBN 0-520-22116-8.
			
			\bibent
			Chen, Tina Mai. 2003. \textit{Female Icons, Feminist Iconography? Socialist Rhetoric and Women’s Agency in 1950s China.} Gender and History, 15(2), 268–295.
			
			\bibent
			Gao, Xiaoxian. 2006. \textit{‘The Silver Flower Contest’: Rural Women in 1950s China 
				and the Gendered Division of Labour.} 18(3). 594–612.
			
			\bibent
			Hershatter, Gail. \textit{Woman and China's Revolution}, 2019. Print.
			
			\bibent
			Ko, Dorothy and Wang, Zheng. 2007. \textit{Translating Feminisms in China}. Oxford: Blackwell Publishing.
			
			\bibent
			Leung, Alicia S. M. 2003. \textit{Feminism in Transition: Chinese Culture, Ideology and the Development of the Women’s Movement in China}. Asia Pacific Journal of Management, 359–374.
			
			\bibent
			Liu, Lydia H., Karl, Rebecca E., and Ko, Dorothy. 2013. \textit{The Birth of Chinese Feminism.} New York: Columbia University Press.
			
			\bibent
			Mann, Susan L. 2011. \textit{Gender and Sexuality in Modern Chinese History.} New York: Cambridge University Press
			
			\bibent
			Min, Dongchao. 2017. \textit{Translation and Travelling Theory: Feminist Theory and Praxis in China.} New York: Routledge.
			
			\bibent Wang, Zheng. 2017. \textit{Finding Women in the State: A Socialist Feminist Revolution in the People’s Republic of China.} Oakland, CA: University of California Press.
			
			\bibent
			Wang, Lihua. 1999. \textit{The Seeds of Socialist Ideology: Women’s Experiences in Beishadao Village.} Women’s Studies International Forum, 22(1).
			
			\bibent Xu, Feng. 2009. \textit{Chinese Feminisms Encounter International Feminisms. } International Feminist Journal of Politics. 11(2), 196–215.
			
			\bibent
			Yu, Zhongli. 2015. \textit{Translating Feminism in China: Gender, Sexuality and Censorship.} London and New York: Routledge.
			
			\bibent
			Yu, Zhongli. 2020. \textit{Translating feminism in China: A historical perspective.} 
			
			\bibent
			Yun, Sung Hyun. 2012. \textit{An analysis of Confucianism’s yin-yang harmony with nature and the traditional oppression of women: Implications for social work practice.} Journal of Social Work, 13(6), 582–598. doi:10.1177/1468017312436445 
			
			
		\end{workscited}
		
	\end{flushleft}
\end{document}
\}