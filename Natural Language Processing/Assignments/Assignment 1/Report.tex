\documentclass[12pt,a4paper]{article}
\usepackage{amsmath}
\usepackage{amssymb}
\usepackage{epstopdf}
\usepackage{inputenc}
\usepackage{graphicx}
\usepackage{titletoc} 
\usepackage{fancyhdr}   
\usepackage[a4paper,pdftex]{geometry}	
\usepackage[english]{babel}
\usepackage{xcolor} 
\usepackage{enumerate}
\usepackage{fix-cm} 
\usepackage[notlof]{tocbibind}
\usepackage{amsmath}
\usepackage{listings}
\usepackage{float}
\usepackage{enumitem}
\usepackage{xcolor}
\usepackage{listings}
\definecolor{vgreen}{RGB}{104,180,104}
\definecolor{vblue}{RGB}{49,49,255}
\definecolor{vorange}{RGB}{255,143,102}
\renewcommand\lstlistingname{Appendix}
\renewcommand\lstlistlistingname{Appendix}

\makeatletter
\newcommand*\@lbracket{[}
\newcommand*\@rbracket{]}
\newcommand*\@colon{:}
\newcommand*\colorIndex{%
	\edef\@temp{\the\lst@token}%
	\ifx\@temp\@lbracket \color{black}%
	\else\ifx\@temp\@rbracket \color{black}%
	\else\ifx\@temp\@colon \color{black}%
	\else \color{vorange}%
	\fi\fi\fi
}
\makeatother

\usepackage{trace}

\usepackage{subcaption}
\begin{document}
	\begin{titlepage}
		\begin{center}
			\includegraphics[scale=.4]{Figures/Cover}\\
			\vspace{1cm}
			\bf{ \large {Department of Computer Science and Technology} }
		\end{center}
		
		\vspace{4cm}
		\centering
		\textbf{\Huge Natural Language Processing}
		\vspace{.5cm}
		
		{\Large Assignment 2}

		\vspace{4cm}
		
		\textbf{\LARGE Sahand Sabour}
		
		
		
		\vspace{0.5cm}
		
		{\large 2020280401}
		
		
		\vfill
		
	\end{titlepage}

	\section{Gradient Calculation}
	Assuming that we are given a predicted word vector $v_c$ for the center
word c in skip-gram, and word prediction is made with the following Softmax function:
	\begin{equation}
		y_o = p(o|c) = \frac{exp(u^T_ov_c)}{\sum_{w=1}^{W}exp(u^T_wv_c)}
	\end{equation}
	where w denotes the w-th word and $u_w= (w = 1, ..., W)$ are the “output” word
vectors for all words in the vocabulary. In addition, assuming that the cross entropy loss is used, we would have:
	\begin{equation}
		\mathcal{L} = - \sum_{w=1}^{W} t_i log(y_i) = - log(p(o|c)) = - log (exp(u^T_ov_c)) + log (\sum_{w=1}^{W}exp(u^T_wv_c))
	\end{equation}
	Where t is the label and is either 1 or 0 since the input is one-hot encoded. Therefore, there would only be one element of the sum that is non-zero (for the expected word o). Further simplifying the loss function gives:
	\begin{equation}
		\mathcal{L} = - u^T_ov_c + log (\sum_{w=1}^{W}exp(u^T_wv_c))
	\end{equation}
	Accordingly, we would derive the gradient from this loss as follows:
	\begin{equation}
		\begin{aligned}
			\frac{\delta \mathcal{L}}{\delta v_c} = -\frac{\delta u^T_ov_c}{\delta v_c} + \frac{\delta log(\sum_{w=1}^{W}exp(u^T_wv_c))}{\delta v_c} \qquad \qquad
			\\
			=-u^T_o + (\frac{1}{\sum_{w=1}^{W}exp(u^T_wv_c)})(exp(u^T_wv_c))(u_w^T)\\  =  \sum_{w=1}^{W} p(o|c) u_w^T - u^T_o\qquad \qquad\qquad \qquad \qquad
		\end{aligned}
	\end{equation}
	
	\section{Word2vec Implementation}
	
	\section{Word2vec Improvement}

	
\end{document}